\section{Gambaran Umum Solusi} \label{sec:gambaran-umum}

Solusi yang diusulkan memodifikasi arsitektur DecMed dengan mengintegrasikan PRE yang diperkuat dengan kontrol akses CapBAC berbasis \textit{Asymmetric Macaroons}. Penelitian ini berfokus pada pembaruan mekanisme otorisasi dan akses darurat.

Secara garis besar, solusi ini terdiri dari tiga poin utama: (1) implementasi \textit{Asymmetric Macaroons} untuk delegasi akses terdesentralisasi, (2) \textit{Server PRE} sebagai validator akses dan transformator kunci, serta (3) \textit{Emergency Oracle} untuk mekanisme \textit{Break-Glass}. Arsitektur sistem secara keseluruhan dapat dilihat pada Lampiran \ref{lampiran:arsitektur-sistem} dan skema penyimpanan \textit{on-chain} dan \textit{off-chain} dapat dilihat pada Lampiran \ref{lampiran:skema-penyimpanan}.

\subsection{Integrasi Macaroons dan Delegasi Akses} \label{subsec:integrasi-macaroons}

Sistem ini menerapkan konsep \textit{Signature-Derived Root} pada Macaroons. \texttt{RootKey} dari setiap token diturunkan secara dinamis dari hasil \textit{Digital Signature} menggunakan kunci privat (\textit{wallet}) pasien terhadap ID token. \textit{Server} PRE dapat memverifikasi macaroon dengan mencocokkan tanda tangan digital terhadap \textit{wallet address} pasien yang terdaftar di jaringan IOTA.

Dokter dapat melakukan delegasi akses kepada pihak lain tanpa interaksi kembali dengan pasien, tentunya dengan kemampuan menambahkan \textit{caveats} secara lokal. Dengan demikian, pasien memegang kendali penuh atas penerbitan akses. Rancangan detail \textit{caveats} dapat dilihat pada Lampiran \ref{lampiran:skema-caveats}.

\subsection{Mekanisme Akses Data RME} \label{subsec:mekanisme-akses-data-rme}

Pengelolaan akses data ditangani oleh \textit{server} PRE yang memvalidasi token Macaroon sebelum melakukan \textit{Re-Encryption}. Tahapan validasi meliputi pengecekan tanda tangan digital dan evaluasi \textit{caveats} yang melekat pada token. Detail operasional dan urutan langkah dalam mekanisme akses data ini dijelaskan pada Lampiran \ref{lampiran:mekanisme-akses-data}.

\subsection{Mekanisme Akses Darurat (\textit{Break-Glass})} \label{subsec:mekanisme-break-glass}

Sistem ini menerapkan mekanisme \textit{Key Recovery} yang terdesentralisasi untuk menjamin ketersediaan data saat pasien tidak sadar, sekaligus mencegah penyalahgunaan akses oleh salah satu pihak.

\subsubsection{Enkapsulasi Kunci Ganda}
\label{subsec:enkapsulasi-kunci-ganda}

Untuk mempersiapkan kondisi darurat sejak awal, aplikasi klien melakukan enkapsulasi kunci simetris (AES) ke dalam dua bentuk kapsul saat data pertama kali diunggah:
\begin{enumerate}
	\item \textbf{Normal Capsule:} Kunci AES dienkripsi menggunakan Kunci Publik Pasien (untuk akses reguler dan delegasi).
	\item \textbf{Emergency Capsule:} Kunci AES yang sama dienkripsi menggunakan \textit{Global Public Key} milik konsorsium \textit{Emergency Oracle}.
\end{enumerate}

\subsubsection{Desentralisasi \textit{Oracle} dengan SSS}
Untuk memitigasi risiko SPoF, \textit{Emergency Oracle} menggunakan skema \textit{Shamir's Secret Sharing} (SSS) dengan konfigurasi \(3\)-of-\(5\). Detail langkah pemulihan akses darurat dapat dilihat pada Lampiran \ref{lampiran:mekanisme-break-glass}.
