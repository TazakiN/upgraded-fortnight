\section{Latar Belakang}
\label{sec:latarbelakang}

Pada era digital ini, sistem rekam medis sudah mulai berkembang dan mulai memanfaatkan teknologi digital untuk membantu proses pengelolaan rekam medis. Hal ini dapat dilihat dari adanya  Peraturan Menteri Kesehatan Republik Indonesia (Permenkes)
Nomor 24 Tahun 2022 tentang Rekam Medis yang mengatur pengelolaan rekam medis di Indonesia \autocite{permenkes24_2022}. Rekam Medis adalah bagian penting dalam keseluruhan sistem kesehatan di Indonesia yang menyimpan data dari pasien, sehingga proses pengelolaan rekam medis sendiri memerlukan perhatian khusus. Permenkes tersebut menyebutkan bahwa setiap fasilitas layanan kesehatan di Indonesia wajib untuk mengimplementasikan sistem Rekam Medis Elektronik (RME) sesuai dengan standar yang ditetapkan oleh Permenkes \autocite{kepmenkes1423_2022}.

% apa mendingan yang bahas masalah keamanan diskip aja?
% Meskipun menawarkan berbagai manfaat potensial, implementasi RME menghadapi tantangan signifikan dalam aspek keamanan yang dapat mengancam kualitas layanan kesehatan dan keamanan data pasien. Terdapat beberapa kejadian kebocoran data dari sektor kesehatan di Amerika \autocite{vankessel2023strengthening} \autocite{hipaajournal2024report}. Indonesia juga tidak luput dari ancaman ini. Pada tahun 2022 dicurigai terjadi kebocoran data pasien Covid-19 yang dimiliki oleh Kementerian Kesehatan, yang ditengarai berasal dari 6 juta rekam medis pasien \autocite{hukumonline_kemenkes}.

Implementasi Rekam Medis Elektronik (RME) di Indonesia menghadapi masalah interoperabilitas. Untuk mengatasinya, Peraturan Menteri Kesehatan Nomor 24 Tahun 2022 tentang Rekam Medis \autocite{permenkes24_2022}, sebagai aturan pelaksana Undang-Undang Nomor 29 Tahun 2004 tentang Praktik Kedokteran \autocite{uu_29_2004}, mengamanatkan adanya integrasi dan interoperabilitas sistem serta data Rekam Medis di Fasilitas Pelayanan Kesehatan dengan Platform SATUSEHAT.

Upaya ini juga didukung oleh penerbitan Permenkes Nomor 18 Tahun 2022 tentang Penyelenggaraan Satu Data Bidang Kesehatan melalui Sistem Informasi Kesehatan \autocite{permenkes18_2022}. SATUSEHAT merupakan sistem informasi kesehatan yang digunakan oleh Kementerian Kesehatan Republik Indonesia untuk mengintegrasikan lebih dari 400 aplikasi RME di rumah sakit di seluruh Indonesia \autocite{martinadewi2024analysis}.

SATUSEHAT sendiri adalah ekosistem digital yang terdiri atas beberapa aplikasi dan produk terkait sistem informasi kesehatan di Indonesia \autocite{kemenkes_satusehat}. Platform SATUSEHAT adalah salah satu komponen dari ekosistem SATUSEHAT yang digunakan untuk mengintegrasikan sistem informasi kesehatan di Indonesia. Data RME dari setiap fasilitas pelayanan kesehatan di Indonesia diintegrasikan melalui platform SATUSEHAT. SATUSEHAT menyediakan dokumentasi resmi mekanisme integrasi RME dengan platform SATUSEHAT \autocite{satusehat_playbook} dan video tutorial lengkap tentang integrasi RME dengan platform SATUSEHAT \autocite{satusehat_video_tutorial}.

SATUSEHAT adalah upaya integrasi sistem informasi kesehatan di Indonesia yang patut untuk didukung. Namun, implementasi ini menghasilkan sistem yang tersentralisasi dan menghasilkan sistem dengan \textit{single point of failure} yang dapat mengancam keamanan data pasien. Sektor kesehatan termasuk ke dalam sektor yang sering mendapatkan ancaman keamanan data di seluruh belahan dunia \autocite{vankessel2023strengthening} \autocite{hipaajournal2024report}, termasuk Indonesia \autocite{hukumonline_kemenkes}. Oleh karena itu, sistem terdesentralisasi berbasis DLT bisa menjadi solusi yang lebih baik untuk mengatasi masalah keamanan dan akses data \autocite{ferreira2024enhancing}.

% Sistem yang murni terdesentralisasi berpotensi lambat untuk dikelola, dipelihara, dan diperbaharui. Hal ini mendorong adanya resentralisasi kekuasaan de facto kepada entitas yang lebih kecil dan bisa menjadi penanggung jawab, dalam hal ini adalah Pemerintah \autocite{bodo2020logics}. Karena itu keterlibatan Pemerintah tidak terhindarkan dalam sistem terdesentralisasi.

Aspek keamanan merupakan salah satu aspek penting dalam sistem informasi kesehatan. Oleh karena itu, solusi sistem terdesentralisasi berbasis DLT juga memerlukan perhatian khusus dalam aspek keamanan, terutama akses kontrol terhadap data \autocite{alhaqbani2008access}. Penelitian terkait menyatakan bahwa belum ada model akses kontrol yang memenuhi persyaratan untuk sistem informasi kesehatan \autocite{alhaqbani2008access}, sehingga diperlukan model akses kontrol yang bisa memberikan akses tergranular dengan pemikiran \textit{patient-centric} \autocite{peng2023patientcentric}.

Sudah terdapat implementasi solusi sistem terdesentralisasi berbasis DLT untuk sistem informasi kesehatan, seperti DecMed yang memanfaatkan DLT IOTA, IPFS, dan Proxy Re-Encryption (PRE). Solusi ini mengimplementasikan capability based access control \autocite{exceline2023security} untuk mengatasi masalah akses kontrol terhadap data RME dengan memanfaatkan mekanisme enkripsi data Proxy Re-Encryption (PRE). Namun, implementasi solusi tersebut memiliki kelemahan dalam hal penyebarluasan akses kepada pihak yang lebih luas dan mekanisme delegasi RME untuk kasus gawat darurat \autocite{yang2018lightweight}.

Pada Tugas Akhir ini diusulkan pengembangan dari DecMed, yaitu sebuah kerangka kerja sistem manajemen RME yang mengembangkan DecMed berbasis model Capability-based Access Control memanfaatkan teknologi Macaroons \autocite{birgisson2014macaroons} untuk mengatasi masalah penyebarluasan akses kepada pihak yang lebih luas dan mekanisme delegasi RME untuk kasus gawat darurat \autocite{yang2018lightweight}. Pengembangan difokuskan kepada mekanisme kontrol akses dalam kerangka kerja tersebut sehingga dapat meningkatkan keamanan dari data RME yang tersimpan dan memberikan kontrol sepenuhnya kepada pasien sebagai pemilik data RME.
