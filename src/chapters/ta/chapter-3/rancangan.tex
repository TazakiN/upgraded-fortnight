\section{Gambaran Umum Solusi} \label{sec:gambaran-umum}

Solusi yang diusulkan memodifikasi arsitektur DecMed dengan mengganti mekanisme kontrol akses berbasis PRE menjadi CapBAC menggunakan token Macaroons. Perubahan ini bertujuan untuk memberikan pasien kendali penuh atas granularitas akses data medis mereka serta memungkinkan penerapan mekanisme akses darurat.

Secara garis besar, solusi ini terdiri dari tiga poin utama: (1) integrasi Macaroons untuk delegasi akses, (2) perubahan komponen PRE menjadi \textit{Access Gateway}, (3) \textit{Emergency Oracle} untuk mekanisme \textit{Break-Glass}. Arsitektur sistem secara keseluruhan dapat dilihat pada Lampiran \ref{lampiran:arsitektur-sistem}. Skema penyimpanan data \textit{on-chain} dan \textit{off-chain} dapat dilihat pada Lampiran \ref{lampiran:skema-penyimpanan}.

\subsection{Integrasi Macaroons dan Delegasi Akses} \label{subsec:integrasi-macaroons}

Berbeda dengan sistem terdahulu di mana delegasi memerlukan pembuatan ulang kunci enkripsi di server, sistem ini memanfaatkan \textit{Offline Attenuation} dari Macaroons. \texttt{RootKey} diturunkan secara deterministik menggunakan  \textit{seed} IOTA pasien, menjamin bahwa pasien adalah satu-satunya entitas yang dapat menerbitkan (\textit{Root Macaroon}).

Pasien dapat melakukan atenuasi secara lokal di perangkat mereka dengan menambahkan \textit{caveats} pada token yang akan dikirimkan. Token kemudian dikirimkan langsung ke perangkat tenaga medis terkait melalui mekanisme pertukaran kunci yang aman (misalnya pemindaian QR Code). Dokter dapat melakukan delegasi akses lagi ke pihak lain yang terlibat tanpa bantuan pasien. Rancangan detail \textit{caveats} dapat dilihat pada Lampiran \ref{lampiran:skema-caveats}. \textit{Sequence diagram} dari skenario delegasi akses dapat dilihat pada Lampiran \ref{lampiran:sequence-diagram-delegasi-akses}.

% Revokasi akses adalah salah satu fitur yang menghilang sebagai \textit{trade-off} dari penghapusan PRE. Masalah ini dapat diselesaikan dengan memanfaatkan sebuah \textit{database} yang terhubung pada \textit{Access Gateway}. \textit{Database} ini berfungsi sebagai \textit{revocation list} yang menyimpan informasi tentang token yang telah direvoke.

\subsection{Mekanisme Akses Data RME} \label{subsec:mekanisme-akses-data-rme}

Fungsi PRE pada DecMed digantikan oleh \textit{Access Gateway}. Komponen ini bertindak sebagai validator akses yang \textit{stateless}. Tugas \textit{Access Gateway} adalah memverifikasi integritas tanda tangan HMAC pada Macaroon dan melakukan \textit{key unwrapping} untuk memberikan kunci dekripsi simetris (AES) jika verifikasi berhasil.

Sequence diagram dari skenario ketika akses menulis data RME dilakukan dapat dilihat pada Lampiran \ref{lampiran:sequence-diagram-akses-write-data-rme}. Sedangkan sequence diagram dari skenario ketika akses membaca data RME dilakukan dapat dilihat pada Lampiran \ref{lampiran:sequence-diagram-akses-read-data-rme}.

\subsection{Mekanisme Akses Darurat (Break-Glass)} \label{subsec:mekanisme-break-glass}

Penanganan situasi gawat darurat difasilitasi melalui mekanisme \textit{Third-Party Caveats}. Dokter dapat menerbitkan "Token Darurat" ($M_{emerg}$) yang memiliki syarat khusus: token hanya valid jika disertai \textit{Discharge Macaroon} dari \textit{Emergency Oracle}.

Alur kerja mekanisme \textit{Break-Glass} adalah sebagai berikut:
\begin{enumerate}
	\item Dokter melakukan lookup \texttt{DID} pasien dengan menggunakan NIK pasien.
	\item Dokter meminta akses darurat ke \textit{Access Gateway} dengan menyertakan \texttt{DID} pasien. Access Gateway membuat ($M_{emerg}$) untuk dokter.
	\item \textit{Emergency Oracle} memverifikasi kredensial dokter dan menuliskan log transaksi ke IOTA. Dokter mendapatkan \textit{Discharge Macaroon} ($D_{key}$) dari \textit{Emergency Oracle}.
	\item Dokter menyatukan $M_{emerg}$ dan $D_{key}$, lalu mengirimkannya ke \textit{Access Gateway}. \textit{Access Gateway} membuka akses data.
\end{enumerate}

Sequence diagram dari skenario ketika akses darurat dilakukan dapat dilihat pada Lampiran \ref{lampiran:sequence-diagram-akses-darurat}.