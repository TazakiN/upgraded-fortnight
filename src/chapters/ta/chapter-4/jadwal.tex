\section{Jadwal}
\label{sec:jadwal}

Pelaksanaan Tugas Akhir ini direncanakan akan berlangsung selama satu semester (kurang lebih 20 minggu). Rincian jadwal kegiatan dapat dilihat pada Tabel \ref{tab:jadwal-kegiatan}.

\begin{longtable}{|p{0.5cm}|p{6.5cm}|p{0.5cm}|p{0.5cm}|p{0.5cm}|p{0.5cm}|p{0.5cm}|}
    \caption{Jadwal Pelaksanaan Tugas Akhir} \label{tab:jadwal-kegiatan}                                                                                                                                                                                                                                           \\
    \hline
    \multirow{2}{*}{\textbf{No}} & \multirow{2}{*}{\textbf{Kegiatan}}                                                                                                            & \multicolumn{5}{c|}{\textbf{Bulan Ke-}}                                                                                         \\ \cline{3-7}
                                 &                                                                                                                                               & \textbf{1}                              & \textbf{2}          & \textbf{3}          & \textbf{4}          & \textbf{5}          \\ \hline
    \endfirsthead
    \hline
    \multirow{2}{*}{\textbf{No}} & \multirow{2}{*}{\textbf{Kegiatan}}                                                                                                            & \multicolumn{5}{c|}{\textbf{Bulan Ke-}}                                                                                         \\ \cline{3-7}
                                 &                                                                                                                                               & \textbf{1}                              & \textbf{2}          & \textbf{3}          & \textbf{4}          & \textbf{5}          \\ \hline
    \endhead

    1                            & Eksplorasi \& Persiapan Lingkungan termasuk eksplorasi \textit{library Asymmetric Macaroons}, \textit{setup node} IOTA, dan konfigurasi IPFS. & \cellcolor{gray!50}                     &                     &                     &                     &                     \\ \hline

    2                            & Pengembangan fungsi \textit{minting} Macaroon (\textit{Signature-Derived Root}), enkripsi PRE, dan mekanisme \textit{Dual Key Encapsulation}. & \cellcolor{gray!50}                     & \cellcolor{gray!50} &                     &                     &                     \\ \hline

    3                            & Implementasi \textit{smart contract} pada IOTA, menyesuaikan dengan skema penyimpanan \textit{on-chain} dan \textit{off-chain}.               &                                         & \cellcolor{gray!50} &                     &                     &                     \\ \hline

    4                            & Penyesuaian \textit{client} agar mampu menangani \textit{Normal Capsule} dan \textit{Emergency Capsule} untuk penyimpanan data medis.         &                                         & \cellcolor{gray!50} & \cellcolor{gray!50} &                     &                     \\ \hline

    5                            & Implementasi PRE sebagai validator dan pengembangan \textit{Emergency Oracle} dengan skema SSS (3-of-5).                                      &                                         &                     & \cellcolor{gray!50} &                     &                     \\ \hline

    6                            & Pengembangan integrasi mekanisme \textit{Break-Glass} antara PRE, \textit{Oracle}, dan IOTA untuk pemulihan akses darurat.                    &                                         &                     & \cellcolor{gray!50} & \cellcolor{gray!50} &                     \\ \hline

    7                            & Integrasi Sistem (\textit{End-to-End}) termasuk penggabungan modul \textit{client}, PRE, \textit{oracle}, IOTA dan IPFS.                      &                                         &                     &                     & \cellcolor{gray!50} &                     \\ \hline

    8                            & Pengujian fungsional, analisis, dan perbaikan sistem.                                                                                         &                                         &                     &                     & \cellcolor{gray!50} & \cellcolor{gray!50} \\ \hline

    9                            & Penulisan Laporan Tugas Akhir.                                                                                                                &                                         &                     &                     & \cellcolor{gray!50} & \cellcolor{gray!50} \\ \hline
\end{longtable}
