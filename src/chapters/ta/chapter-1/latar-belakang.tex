\section{Latar Belakang}
\label{sec:latarbelakang}

Pada era digital ini, transformasi sistem rekam medis didorong secara kuat oleh regulasi pemerintah. Peraturan Menteri Kesehatan Republik Indonesia (Permenkes) Nomor 24 Tahun 2022 mewajibkan fasilitas layanan kesehatan (fasyankes) untuk mengimplementasikan Rekam Medis Elektronik (RME) yang terintegrasi \autocite{permenkes24_2022}. Upaya ini diperkuat dengan Keputusan Menteri Kesehatan HK.01.07/MENKES/1423/2022 tentang standar variabel RME \autocite{kemenkes1423-2022} dan inisiatif platform SATUSEHAT yang bertujuan mengintegrasikan lebih dari 400 aplikasi RME di seluruh Indonesia \autocite{martinadewi2024analysis}. Integrasi ini krusial untuk menjamin interoperabilitas data kesehatan nasional sesuai amanat Undang-Undang Nomor 29 Tahun 2004 \autocite{uu_29_2004}.

Meskipun integrasi melalui SATUSEHAT patut didukung, arsitektur yang tersentralisasi menciptakan risiko keamanan inheren berupa \textit{Single Point of Failure} (SPoF). Pemusatan data dalam skala masif meningkatkan dampak fatal apabila terjadi kegagalan sistem atau peretasan. Sektor kesehatan tercatat sebagai sektor yang rentan terhadap serangan siber secara global \autocite{hipaajournal2024report}. Indonesia tidak luput dari ancaman ini, sebagaimana terindikasi dari dugaan kebocoran jutaan data rekam medis pasien yang dikelola Kementerian Kesehatan pada tahun 2022 \autocite{hukumonline_kemenkes}. Risiko ini memicu kebutuhan akan arsitektur alternatif yang lebih tangguh, yaitu sistem terdesentralisasi berbasis \textit{Distributed Ledger Technology} (DLT) yang dapat menghilangkan SPoF dan mengembalikan kedaulatan data kepada pasien, sekaligus mempermudah pengembangan sistem dengan interoperabilitas data yang tinggi \autocite{ferreira2024enhancing}.

Namun, penerapan sistem informasi kesehatan memiliki tantangan tersendiri, terutama pada aspek kontrol akses \autocite{rele2023securing}. Tantangan utamanya bukan hanya menciptakan sistem yang aman, tetapi juga menyeimbangkan keamanan ketat dengan efisiensi alur kerja klinis. Penelitian menunjukkan bahwa mekanisme keamanan yang terlalu kaku berpotensi menghambat tenaga medis dalam memberikan perawatan cepat, sehingga memicu frustrasi atau penundaan layanan medis yang krusial. Selain itu, kontrol akses harus dapat menjamin kerahasiaan data pasien dari penggunaan sekunder yang tidak sah, seperti komersialisasi atau penelitian tanpa persetujuan.

Regulasi melalui Permenkes No. 24 Tahun 2022 Pasal 26 dan 29 secara spesifik menuntut diimplementasikannya sebuah akses kontrol demi pendukung keamanan dan perlindungan RME \autocite{permenkes24_2022}. Namun, kontrol akses RME konvensional seringkali hanya membedakan peran kasar seperti 'dokter' atau 'perawat', padahal kebutuhan lapangan menuntut kontrol yang bersifat \textit{fine-grained} terkait kontrol data pasien. Selain terkait akses pada kondisi normal, sistem RME juga perlu mekanisme akses darurat pada kondisi pasien tidak sadar atau dalam keadaan darurat yang membahayakan nyawa \autocite{permenkes24_2022}. Dalam keadaan darurat, akses darurat perlu dapat dilakukan dengan cepat, tanpa melalui prosedur otentikasi yang berbelit. Namun akses tersebut tetap tercatat secara \textit{immutable} untuk kebutuhan \textit{audit} \autocite{alamin2025did}. Oleh karena itu, dibutuhkan mekanisme kontrol akses yang mampu menegakkan batasan granular pada kondisi normal, namun tetap efisien dan \textit{auditable} saat kondisi darurat medis terjadi.

Model akses kontrol konvensional dinilai memiliki kekurangan untuk mengatasi kompleksitas ini \autocite{singh2025granular}. Model \textit{Role-Based Access Control} (RBAC), meskipun menjadi standar industri, dianggap terlalu statis dan kasar (\textit{coarse-grained}) karena memberikan akses luas berdasarkan jabatan tanpa mempertimbangkan konteks hubungan dokter-pasien spesifik. Meskipun model \textit{Attribute-Based Access Control} (ABAC) menawarkan granularitas lebih baik, implementasinya pada lingkungan terdesentralisasi membebani kinerja sistem akibat evaluasi kebijakan yang kompleks di setiap permintaan akses.

Beberapa solusi terdesentralisasi telah diusulkan, salah satunya adalah DecMed. Sistem ini memanfaatkan DLT IOTA, IPFS, dan \textit{Proxy Re-Encryption} (PRE) dengan menerapkan \textit{capability-based access control}.
% \autocite{exceline2023security} 
Meskipun DecMed berhasil mendelegasikan hak akses dari pasien kepada tenaga medis, mekanisme token kapabilitas pada DecMed bersifat statis. Proses delegasi akses ke pihak ketiga memerlukan penerbitan token baru oleh pasien. Hal ini tidak efisien untuk skenario delegasi berantai di lingkungan medis yang terdiri dari beberapa tenaga medis. Lebih jauh lagi, sistem ini belum optimal dalam menangani akses darurat (\textit{break-glass}) yang membutuhkan respons cepat tanpa prosedur otentikasi yang berbelit, namun tetap harus tercatat secara \textit{immutable} untuk kebutuhan \textit{audit}.

Berdasarkan permasalahan tersebut, Tugas Akhir ini mengusulkan pengembangan kerangka kerja sistem manajemen RME berbasis DecMed dengan memperbaharui mekanisme kontrol aksesnya menggunakan teknologi Macaroons. Kebaruan utama penelitian ini terletak pada pemanfaatan fitur \textit{attenuation} pada Macaroons \autocite{birgisson2014macaroons}. Berbeda dengan mekanisme yang diimplementasikan pada sistem DecMed, Macaroons memungkinkan pemegang akses untuk mendelegasikan haknya secara terbatas kepada pihak lain tanpa perlu berinteraksi kembali dengan pasien, serta mendukung verifikasi \textit{caveats} kontekstual untuk skenario \textit{break-glass} yang \textit{tamper-evident}. Pengembangan ini difokuskan untuk mengatasi kelemahan mekanisme delegasi dan akses darurat pada penelitian sebelumnya, sehingga dapat meningkatkan keamanan data RME sekaligus memberikan kontrol akses yang utuh dan granular kepada pasien.