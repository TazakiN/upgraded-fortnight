\section{Break-Glass}
\label{sec:break-glass}

\textit{Break-glass} adalah mekanisme darurat yang memungkinkan pengguna dengan akses terbatas untuk mendapatkan akses terhadap informasi penting dalam situasi kritis, tanpa menunggu proses otorisasi standar \autocite{alamin2025did}. Dalam konteks healthcare, \textit{break-glass} adalah prosedur yang memungkinkan profesional medis untuk mengakses data rekam medis pasien tanpa persetujuan pasien, tetapi hanya dalam keadaan darurat medis yang mengancam jiwa. Mekanisme ini berakar pada prinsip layanan kesehatan bahwa perawatan pasien lebih diutamakan daripada pembatasan kontrol akses dalam situasi yang mengancam jiwa \autocite{yeng2022analysing}.

Konsep akses \textit{break-glass} berasal dari mekanisme keamanan fisik di mana kaca harus dipecahkan untuk mengakses peralatan darurat, meninggalkan bukti penggunaan yang terlihat \autocite{NEMA2004BreakGlass}. Dalam sistem layanan kesehatan digital, ini diterjemahkan menjadi mekanisme yang: (1) memberikan akses segera kepada personel darurat yang berwenang, (2) membuat jejak audit yang tidak dapat diubah, dan (3) memicu proses peninjauan pasca-kejadian.

% Beberapa pendekatan arsitektur untuk mekanisme \textit{break-glass} telah diusulkan. (\textit{Lightweight Break-Glass Access Control/LiBAC}) mengimplementasikan mode akses ganda: akses berbasis atribut untuk situasi normal dan akses \textit{break-glass} yang melewati kebijakan akses selama keadaan darurat \autocite{yang2018lightweight}. Sistem ini terbukti aman secara formal dalam model standar sambil mempertahankan efisiensi yang sesuai untuk layanan kesehatan.

