\chapter{Mekanisme Akses Darurat (\textit{Break-Glass})}
\label{lampiran:mekanisme-break-glass}

\begin{enumerate}
	\item Dokter mengajukan permintaan darurat ke jaringan \textit{Oracle}.
	\item Setiap \textit{node Oracle} memverifikasi kredensial dokter dan mencatat persetujuan (\textit{partial discharge}) jika valid.
	\item Sistem memerlukan persetujuan minimal dari 3 \textit{node} ($t=3$) untuk dapat merekonstruksi kunci privat sementara.
	\item Setelah ambang batas tercapai, \textit{Server PRE} menggabungkan pecahan tersebut untuk mendekripsi \textit{Emergency Capsule}, lalu segera melakukan re-enkripsi untuk dokter pemohon.
\end{enumerate}

Detail urutan langkah dalam mekanisme akses darurat ini dapat dilihat pada Gambar \ref{fig:seq-break-glass}.

\begin{figure}[ht]
	\centering
	\includegraphics[width=\textwidth]{resources/appendix/sequence_break_glass.png}
	\caption{\textit{Sequence Diagram} Alur Akses Darurat (\textit{Threshold Oracle 3-of-5})}
	\label{fig:seq-break-glass}
\end{figure}