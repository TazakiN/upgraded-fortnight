\chapter{Mekanisme Akses Data RME}
\label{lampiran:mekanisme-akses-data}

Komponen \textit{Proxy Re-Encryption} (PRE) bertindak sebagai satu-satunya titik masuk untuk permintaan data medis. Tahapan kerja PRE adalah sebagai berikut:

\begin{enumerate}
	\item Menerima permintaan akses beserta token Macaroon.
	\item Melakukan resolusi identitas pasien dari ID token untuk mendapatkan kunci publik atau alamat dompet yang sesuai.
	\item Mengevaluasi seluruh (\textit{caveats}) yang melekat pada token
	\item Memverifikasi validitas tanda tangan kriptografis pada token.
	\item Menjalankan fungsi \textit{Re-Encryption} untuk mengubah enkripsi data pasien agar dapat diakses oleh dokter, hanya jika validasi token berhasil.
\end{enumerate}

Detail urutan langkah dalam mekanisme akses data ini dapat dilihat pada Gambar \ref{fig:seq-akses-normal}.

\begin{figure}[ht]
	\centering
	\includegraphics[width=\textwidth]{resources/appendix/sequence_akses_normal.png}
	\caption{\textit{Sequence Diagram} Alur Akses Data RME}
	\label{fig:seq-akses-normal}
\end{figure}

