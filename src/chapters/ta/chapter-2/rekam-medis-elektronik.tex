\section{Rekam Medis Elektronik}

\subsection{Definisi RME}
\label{subsec:definisi-rme}

\textit{World Health Organization} (\textit{WHO}) mendefinisikan Rekam Medis Elektronik (\textit{RME}) sebagai sebuah catatan kesehatan \textit{waktu nyata} yang memiliki akses terhadap bukti-bukti terkait yang bisa membantu proses pengambilan keputusan dalam usaha untuk menyembuhkan seorang pasien \autocite{who2012management}.

Di Indonesia, definisi dari Rekam Medis dan Rekam Medis Elektronik didefinisikan pada Peraturan Menteri Kesehatan Republik Indonesia Nomor 24 Tahun 2022, "Rekam Medis adalah dokumen yang berisikan identitas pasien, pemeriksaan, pengobatan, tindakan, dan pelayanan lain yang telah diberikan pada pasien" sedangkan "Rekam Medis Elektronik adalah Rekam Medis yang dibuat dengan menggunakan \textit{sistem elektronik} yang diperuntukkan bagi penyelenggaraan Rekam Medis." \autocite{permenkes24_2022}.

Berdasarkan definisi tersebut, rekam medis adalah salah satu komponen vital dalam \textit{sistem} kesehatan di Indonesia. Selain itu, rekam medis juga bukanlah sebuah catatan biasa yang dibuat lalu disimpan, melainkan sebuah catatan yang terus diperbaharui dari awal hingga akhir pengobatan. Keberadaan Rekam Medis bertujuan untuk meningkatkan mutu pelayanan kesehatan dan memberikan kepastian hukum dalam peristiwa medis.

\subsection{Isi RME}
\label{subsec:isi-rme}

Berdasarkan Peraturan Menteri Kesehatan Republik Indonesia Nomor 24 Tahun 2022 Pasal (27) ayat (1), isi dari Rekam Medis Elektronik adalah sebagai berikut:

\begin{enumerate}
	\item Dokumentasi Administratif
	\item Dokumentasi Klinis
\end{enumerate}

Adapun Fasilitas Pelayanan Kesehatan (\textit{Fasyankes}) dapat mengembangkan isi Rekam Medis Elektronik sebagaimana dimaksud pada ayat (1) sesuai dengan kebutuhan pelayanan kesehatan \autocite{permenkes24_2022}.

Untuk memberikan panduan yang lebih rinci terkait isi dari Rekam Medis Elektronik, Kementerian Kesehatan Republik Indonesia mengeluarkan Keputusan Menteri Kesehatan Nomor HK.01.07/MENKES/1423/2022 tentang Pedoman \textit{Variabel} dan \textit{Meta Data} pada Penyelenggaraan Rekam Medis Elektronik \autocite{kepmenkes1423_2022}. Keputusan ini mengatur \textit{variabel} dan \textit{metadata} yang harus ada dalam \textit{sistem RME}.

% kalo ditampilin keknya mirip banget sama punya ka hari

\subsection{Kepemilikian dan Hak Akses RME}
\label{subsec:kepemilikian-hak-akses-rme}

Berdasarkan Peraturan Menteri Kesehatan Republik Indonesia Nomor 24 Tahun 2022 Pasal (25) ayat (1), Dokumen dari Rekam Medis Elektronik adalah milik dari \textit{Fasyankes}. \textit{Fasyankes} bertanggung jawab atas dokumen Rekam Medis tersebut. \autocite{permenkes24_2022}.

Pasal selanjutnya yaitu Pasal (26) ayat (1), menyatakan bahwa Isi dari Rekam Medis Elektronik adalah milik dari pasien \autocite{permenkes24_2022}. Sehingga segala hal terkait penyampaian isi rekam medis hanya bisa dilakukan atas persetujuan pasien. Tenaga medis bisa memiliki akses terhadap rekam medis pasien yang tersimpan, untuk melakukan beberapa \textit{aksi} yaitu:

\begin{enumerate}
	\item Menambahkan data baru.
	\item Melakukan perbaikan data.
	\item Melihat Data.
\end{enumerate}

Berdasarkan pasal (32), semua pihak memiliki kewajiban untuk melindungi kerahasiaan dari data rekam medis. pihak yang dimaksud pada ayat ini adalah:

\begin{enumerate}
	\item Tenaga Kesehatan pemberi pelayanan kesehatan, dokter dan dokter gigi, dan/atau Tenaga Kesehatan lain yang memiliki akses terhadap data dan \textit{informasi} kesehatan Pasien.
	\item Pimpinan Fasilitas Pelayanan Kesehatan.
	\item Tenaga yang berkaitan dengan pembiayaan pelayanan kesehatan.
	\item Badan hukum/korporasi dan/atau Fasilitas Pelayanan Kesehatan.
	\item Mahasiswa/siswa yang bertugas dalam pemeriksaan, pengobatan, perawatan, dan/atau \textit{manajemen informasi} di Fasilitas Pelayanan Kesehatan.
	\item Pihak lain yang memiliki akses terhadap data dan \textit{informasi} kesehatan Pasien di Fasilitas Pelayanan Kesehatan.
\end{enumerate}

\subsection{Ancaman Siber terhadap RME}
\label{subsec:ancaman-terhadap-rme}

Ancaman Siber didefinisikan sebagai semua keadaan, kejadian, atau aksi --- yang seringkali diterhadap Rekam Medis Elektronik dapat dikategorikan menjadi beberapa jenis, antara lain:

\begin{enumerate}
	\item Ancaman dari luar.
	\item Ancaman dari dalam.
\end{enumerate}

Ancaman dari luar dapat dikategorikan menjadi beberapa jenis, antara lain: