% \chapter{Daftar Variabel dan Metadata Rekam Medis Elektronik}
% \label{lampiran:metadata-rme}

% \section*{A. Kelompok Data Administratif dan Identitas (Berlaku Umum)}
% Variabel ini digunakan pada semua titik layanan (IGD, Rawat Jalan, Rawat Inap):
% \begin{itemize}
% 	\item \textbf{Identitas Pasien:} Nama Lengkap, No. Rekam Medis, NIK, No. Paspor/KITAS (WNA), Nama Ibu Kandung, Tempat/Tanggal Lahir, Jenis Kelamin, Agama, Suku, Bahasa yang Dikuasai.
% 	\item \textbf{Alamat dan Kontak:} Alamat Lengkap (KTP \& Domisili), RT/RW, Kelurahan, Kecamatan, Kota/Kab, Provinsi, Kode Pos, Negara, No. Telepon Rumah/Seluler.
% 	\item \textbf{Data Sosial Ekonomi:} Pendidikan Terakhir, Pekerjaan, Status Pernikahan.
% 	\item \textbf{Penanggung Jawab:} Nama, Nomor Telepon, Hubungan dengan Pasien.
% 	\item \textbf{Cara Pembayaran:} JKN, Mandiri, atau Asuransi Lainnya.
% \end{itemize}

% \section*{B. Instalasi Gawat Darurat (IGD)}
% \begin{enumerate}
% 	\item \textbf{Triase dan Penilaian Awal:}
% 	      \begin{itemize}
% 		      \item Tanggal \& Jam Masuk, Sarana Transportasi Kedatangan.
% 		      \item Kondisi Pasien Tiba (Death on Arrival, Resusitasi, Emergency, Urgent, dll).
% 		      \item Keluhan Utama, Riwayat Penyakit, Riwayat Alergi (Obat/Makanan/Udara), Riwayat Pengobatan.
% 	      \end{itemize}

% 	\item \textbf{Asesmen Nyeri dan Risiko Jatuh:}
% 	      \begin{itemize}
% 		      \item Skala Nyeri (VAS/NRS/Wong Baker), Lokasi, Durasi, Frekuensi.
% 		      \item Risiko Jatuh (Morse Fall Scale / Humpty Dumpty).
% 	      \end{itemize}

% 	\item \textbf{Pemeriksaan Fisik:}
% 	      \begin{itemize}
% 		      \item Tanda-tanda Vital (GCS, Denyut Jantung, Pernapasan, Tekanan Darah, Suhu Tubuh).
% 		      \item Pemeriksaan Head-to-Toe (Kepala, Mata, THT, Dada/Thorax, Abdomen/Perut, Ekstremitas, Genital).
% 	      \end{itemize}

% 	\item \textbf{Tindakan dan Medis:}
% 	      \begin{itemize}
% 		      \item Diagnosis (Awal, Kerja, Banding) menggunakan kode ICD-10.
% 		      \item Tindakan Medis (Nama Tindakan, Waktu Pelaksanaan, Operator) menggunakan ICD-9 CM.
% 		      \item \textit{General Consent} (Persetujuan Umum) dan \textit{Informed Consent}.
% 	      \end{itemize}
% \end{enumerate}

% \section*{C. Rawat Jalan dan Rawat Inap}
% Variabel pada rawat jalan dan inap memiliki kemiripan dengan IGD namun dengan penambahan spesifik:
% \begin{enumerate}
% 	\item \textbf{Asesmen Awal Medis \& Keperawatan:}
% 	      \begin{itemize}
% 		      \item Anamnesis Lanjutan (Keluhan, Riwayat Penyakit, Alergi).
% 		      \item Pemeriksaan Fisik Lengkap (Anatomi Tubuh).
% 		      \item Status Psikologis, Sosial, Ekonomi, dan Spiritual.
% 		      \item Status Fungsional (Alat bantu, Cacat tubuh).
% 		      \item Skrining Gizi dan Skrining Nyeri.
% 	      \end{itemize}

% 	\item \textbf{Perencanaan dan Instruksi (CPPT):}
% 	      \begin{itemize}
% 		      \item Rencana Rawat (\textit{Care Plan}).
% 		      \item Instruksi Medik dan Keperawatan.
% 		      \item Perencanaan Pemulangan Pasien (\textit{Discharge Planning}) -- \textit{Khusus Rawat Inap}.
% 	      \end{itemize}

% 	\item \textbf{Diagnosis dan Terapi:}
% 	      \begin{itemize}
% 		      \item Diagnosis Primer dan Sekunder (ICD-10).
% 		      \item Tindakan/Prosedur (ICD-9 CM).
% 		      \item Obat/BMHP yang digunakan.
% 	      \end{itemize}
% \end{enumerate}

% \section*{D. Laboratorium dan Radiologi (Pemeriksaan Penunjang)}
% \begin{enumerate}
% 	\item \textbf{Permintaan Pemeriksaan:}
% 	      \begin{itemize}
% 		      \item Identitas Pengirim (Dokter/Unit), Diagnosis Klinis, Prioritas (Cito/Non-Cito).
% 		      \item Jenis Pemeriksaan dan Spesimen (Darah, Urin, dll).
% 	      \end{itemize}
% 	\item \textbf{Hasil Pemeriksaan:}
% 	      \begin{itemize}
% 		      \item Nilai Hasil, Nilai Rujukan, Interpretasi Hasil, Nilai Kritis.
% 		      \item Tanggal/Jam Validasi dan Nama Dokter Pemeriksa.
% 		      \item Berkas Hasil (Gambar Radiologi/DICOM, PDF Lab).
% 	      \end{itemize}
% \end{enumerate}

% \section*{E. Resep dan Obat (Apotek)}
% \begin{enumerate}
% 	\item \textbf{Data Resep:}
% 	      \begin{itemize}
% 		      \item Identitas Dokter Penulis, Tanggal Penulisan.
% 		      \item Rincian Obat: Nama Obat, Bentuk Sediaan, Dosis, Rute Pemberian (Oral/IV/dll), Frekuensi, Jumlah.
% 	      \end{itemize}
% 	\item \textbf{Telaah Resep:}
% 	      \begin{itemize}
% 		      \item Kajian Administratif, Farmasetik, dan Klinis (Alergi, Interaksi Obat).
% 	      \end{itemize}
% 	\item \textbf{Dispensing:}
% 	      \begin{itemize}
% 		      \item Waktu Penyiapan, Waktu Penyerahan, Petugas Penyerah.
% 	      \end{itemize}
% \end{enumerate}

\chapter{Kamus Data Rekam Medis Elektronik}
\label{lampiran:metadata-rme}

Tabel berikut menjabarkan metadata variabel Rekam Medis Elektronik yang diadopsi dari KMK No. HK.01.07/MENKES/1423/2022 \autocite{kemenkes1423-2022}.

\begin{longtable}{|p{4cm}|p{5cm}|p{5cm}|}
	\caption{Metadata Variabel RME} \label{tab:metadata-rme}                                                    \\
	\hline
	\textbf{Kelompok Data} & \textbf{Variabel Utama}   & \textbf{Keterangan/Format}                             \\ \hline
	\endfirsthead
	\hline
	\textbf{Kelompok Data} & \textbf{Variabel Utama}   & \textbf{Keterangan/Format}                             \\ \hline
	\endhead

	% Identitas
	\multirow{5}{4cm}{Identitas Pasien}
	                       & Nama Lengkap              & Teks (Sesuai KTP)                                      \\ \cline{2-3}
	                       & NIK                       & Numerik (16 digit)                                     \\ \cline{2-3}
	                       & No. Rekam Medis           & Karakter (Unik per RS)                                 \\ \cline{2-3}
	                       & Tgl Lahir / Jenis Kelamin & DD/MM/YYYY / Laki-laki atau Perempuan                  \\ \cline{2-3}
	                       & Nama Ibu Kandung          & Teks                                                   \\ \hline

	% Anamnesis
	\multirow{4}{4cm}{Anamnesis (IGD/RJ/RI)}
	                       & Keluhan Utama             & Free text                                              \\ \cline{2-3}
	                       & Riwayat Penyakit          & Free text (Penyakit terdahulu/keluarga)                \\ \cline{2-3}
	                       & Riwayat Alergi            & Kategori (Obat/Makanan/Udara)                          \\ \cline{2-3}
	                       & Riwayat Pengobatan        & Obat yang sedang dikonsumsi                            \\ \hline

	% Pemeriksaan Fisik
	\multirow{4}{4cm}{Pemeriksaan Fisik}
	                       & Tanda Vital               & TD (mmHg), Nadi (x/mnt), Suhu (C), Nafas (x/mnt)       \\ \cline{2-3}
	                       & Tingkat Kesadaran         & GCS / EWS (Alert, Voice, Pain, Unresponsive)           \\ \cline{2-3}
	                       & Anatomi Tubuh             & Kepala, Thorax, Abdomen, Ekstremitas (Normal/Kelainan) \\ \cline{2-3}
	                       & Skala Nyeri               & VAS / NRS (0-10)                                       \\ \hline

	% Diagnosis & Tindakan
	\multirow{3}{4cm}{Diagnosis \& Tindakan}
	                       & Diagnosis (Awal/Akhir)    & Kode ICD-10                                            \\ \cline{2-3}
	                       & Tindakan Medis            & Kode ICD-9 CM                                          \\ \cline{2-3}
	                       & Informed Consent          & Persetujuan/Penolakan Tindakan                         \\ \hline

	% Resep
	\multirow{4}{4cm}{Peresepan (E-Resep)}
	                       & Item Obat                 & Nama Obat, Kekuatan Sediaan                            \\ \cline{2-3}
	                       & Signa / Aturan Pakai      & Dosis, Rute, Frekuensi (mis: 3x1)                      \\ \cline{2-3}
	                       & Telaah Resep              & Administrasi, Farmasetik, Klinis                       \\ \cline{2-3}
	                       & Status Resep              & Pending, Processed, Delivered                          \\ \hline

	% Lab / Rad
	\multirow{3}{4cm}{Pemeriksaan Penunjang}
	                       & Permintaan                & Jenis Pemeriksaan, Cito/Non-Cito                       \\ \cline{2-3}
	                       & Hasil                     & Nilai, Satuan, Nilai Rujukan                           \\ \cline{2-3}
	                       & Interpretasi              & Bacaan Dokter Spesialis                                \\ \hline
\end{longtable}