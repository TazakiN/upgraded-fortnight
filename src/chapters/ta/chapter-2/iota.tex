\section{IOTA}
\label{sec:iota}

\subsection{DLT}
\label{subsec:dlt}

\textit{Distributed Ledger Technology} adalah sebuah paradigma perubahan dalam cara data direkam, disimpan, dan diverifikasi di beberapa lokasi tanpa kontrol terpusat. \textit{DLT} mencakup berbagai implementasi termasuk \textit{blockchain} dan struktur alternatif seperti \textit{Directed Acyclic Graphs} (\textit{DAGs}), setiap implementasi menawarkan pendekatan yang berbeda untuk mencapai konsensus, keamanan, dan skalabilitas dalam \textit{jaringan} terdesentralisasi \autocite{savadatti2025survey}.

\textit{Sistem DLT} merevolusi komputasi yang aman dan skalabel melalui beberapa prinsip utama. Mereka menyediakan penyimpanan data terdesentralisasi di mana \textit{informasi} direplikasi di banyak \textit{node}, sehingga menghilangkan \textit{single point of failure} \autocite{maza2025distributed}. Karakteristik \textit{immutability} memastikan bahwa setelah data dicatat, data tersebut menjadi tahan terhadap upaya perubahan melalui \textit{hashing} kriptografi dan mekanisme konsensus. Transparansi memungkinkan semua partisipan \textit{jaringan} untuk memverifikasi transaksi sembari tetap menjaga kontrol privasi yang sesuai. \textit{Smart contract} (\textit{kontrak pintar}) memungkinkan perjanjian yang dieksekusi secara otomatis untuk menegakkan aturan yang telah ditentukan tanpa perantara, sehingga mengurangi biaya operasional dan meningkatkan efisiensi \autocite{mohammed2025blockchain}.

\subsection{Definisi IOTA}

\textit{IOTA} merupakan \textit{DLT} khusus yang dirancang secara spesifik untuk \textit{jaringan} dan aplikasi \textit{Internet-of-Things} (\textit{IoT}). Berbeda dengan arsitektur \textit{blockchain} tradisional, \textit{IOTA} menggunakan struktur \textit{Directed Acyclic Graph} (\textit{DAG}) yang disebut \textit{Tangle} sebagai buku besar yang terdistribusi \autocite{sealey2022iota}. Gambar \ref{fig:tangle} menunjukkan contoh dari struktur \textit{Tangle}.

\begin{figure}[ht]
	\centering
	\includegraphics[width=0.6\textwidth]{chapter-2/The-structure-of-the-tangle.png}
	\caption{Struktur \textit{Tangle} pada \textit{IOTA} \autocite{chandel2020using}}
	\label{fig:tangle}
\end{figure}

\textit{IOTA Rebased} merupakan perubahan arsitektur fundamental dari visi \textit{IOTA 2.0} berbasis \textit{DAG} ke \textit{sistem} berbasis \textit{blockchain} yang dibangun pada dasar teknologi yang berbeda. Perubahan ini menangani batasan desentralisasi dan skalabilitas yang menghambat pertumbuhan \textit{ekosistem} dan penerapan \textit{IoT} praktis dari arsitektur sebelumnya \autocite{sedi2025from}. Dengan \textit{IOTA Rebased}, \textit{sistem blockchain} menjadi lebih \textit{scalable} dan desentralisasi, meningkatkan keandalan dan kemampuan untuk menangani peningkatan beban transaksi.
