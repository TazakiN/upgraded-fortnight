\section{\textit{InterPlanetary \textit{File} System} (IPFS)}
\label{sec:ipfs}

IPFS adalah sebuah protokol \textit{peer-to-peer} dan sistem \textit{file} terdistribusi yang dirancang untuk menyimpan dan membagikan data secara desentralisasi \autocite{benet2014ipfs}. Berbeda dengan sistem penyimpanan tradisional yang bergantung pada \textit{server} pusat, IPFS membangun jaringan global dengan setiap pengguna (\textit{node}) dapat berperan sebagai penyimpan dan penyedia konten. Teknologi ini menggunakan \textit{content addressing} daripada \textit{location addressing}, sehingga setiap \textit{file} diidentifikasi dengan unik berdasarkan hash kriptografi dari konten tersebut. Dengan pendekatan desentralisasi ini, IPFS menawarkan solusi yang lebih tahan terhadap sensor, lebih andal, dan tidak bergantung pada satu titik kegagalan pusat.

IPFS beroperasi melalui mekanisme yang mengintegrasikan beberapa komponen kunci. Pertama, menggunakan \textit{Distributed Hash Table} (DHT) yang memungkinkan setiap \textit{file} diberi identitas unik berdasarkan hash konten (\textit{Content Identifier}/CID), memungkinkan \textit{file} diakses dari \textit{node} terdekat yang menyimpannya daripada harus mengakses \textit{server} pusat yang jauh. Kedua, IPFS menerapkan penyimpanan terdistribusi dengan membagi \textit{file} ke dalam blok-blok kecil yang tersebar di berbagai \textit{node} dalam jaringan, sehingga meningkatkan kecepatan akses dan ketahanan terhadap kegagalan. Ketiga, teknologi ini menggunakan sistem \textit{caching} lokal pada setiap \textit{node} yang menyimpan salinan konten yang sering diakses, memungkinkan akses lebih cepat tanpa perlu mengunduh ulang dari internet. Keempat, IPFS menggunakan protokol komunikasi \textit{peer-to-peer} yang memungkinkan setiap \textit{node} bertindak sebagai klien, \textit{server}, atau \textit{node} perantara, sehingga memfasilitasi komunikasi langsung antar \textit{node} tanpa perantara \textit{server} pusat \autocite{benet2014ipfs}.