\chapter{Rancangan Skema Caveats Macaroons}
\label{lampiran:skema-caveats}

\begin{longtable}{|c|>{\raggedright\arraybackslash}p{4cm}|>{\raggedright\arraybackslash}p{8cm}|}
    \caption{Rancangan Skema Caveats Macaroons} \label{tab:skema-caveats}                                                                                           \\
    \hline
    \textbf{No} & \textbf{Caveats}            & \textbf{Keterangan}                                                                                                 \\
    \hline
    \endfirsthead
    \hline
    \textbf{No} & \textbf{Caveats}            & \textbf{Keterangan}                                                                                                 \\
    \hline
    \endhead
    1           & \texttt{expiration\_time}   & Batas waktu \textit{macaroons} tersebut valid.                                                                      \\ \hline
    2           & \texttt{access\_data\_type} & Tipe data yang dapat diakses, contohnya \texttt{ADMINISTRATIVE} atau \texttt{MEDICAL}.                              \\ \hline
    3           & \texttt{access\_rights}     & Hak akses harus eksplisit disebutkan dengan nilainya antara \texttt{WRITE}, \texttt{READ}, dan \texttt{CORRECTION}. \\ \hline
\end{longtable}

