\section{Jadwal}
\label{sec:jadwal}

Pelaksanaan Tugas Akhir ini direncanakan akan berlangsung selama satu semester (kurang lebih 16 minggu). Rincian jadwal kegiatan dapat dilihat pada Tabel \ref{tab:jadwal-kegiatan}.

\begin{longtable}{|p{0.5cm}|p{6.5cm}|p{0.5cm}|p{0.5cm}|p{0.5cm}|p{0.5cm}|}
    \caption{Jadwal Pelaksanaan Tugas Akhir} \label{tab:jadwal-kegiatan}                                                                                                                                                                                                                      \\
    \hline
    \multirow{2}{*}{\textbf{No}} & \multirow{2}{*}{\textbf{Kegiatan}}                                                                                                             & \multicolumn{4}{c|}{\textbf{Bulan Ke-}}                                                                   \\ \cline{3-6}
                                 &                                                                                                                                                & \textbf{1}                              & \textbf{2}          & \textbf{3}          & \textbf{4}          \\ \hline
    \endfirsthead
    \hline
    \multirow{2}{*}{\textbf{No}} & \multirow{2}{*}{\textbf{Kegiatan}}                                                                                                             & \multicolumn{4}{c|}{\textbf{Bulan Ke-}}                                                                   \\ \cline{3-6}
                                 &                                                                                                                                                & \textbf{1}                              & \textbf{2}          & \textbf{3}          & \textbf{4}          \\ \hline
    \endhead

    1                            & Eksplorasi \& Persiapan Lingkungan termasuk eksplorasi library Macaroons, setup node IOTA, dan konfigurasi IPFS.                               & \cellcolor{gray!50}                     &                     &                     &                     \\ \hline

    2                            & Pengembangan fungsi \textit{minting} Macaroon, validasi \textit{caveat}, enkripsi AES-256, dan mekanisme \textit{Key Wrapping}.                & \cellcolor{gray!50}                     & \cellcolor{gray!50} &                     &                     \\ \hline

    3                            & Implementasi API Gateway untuk \textit{issuance} token dan pengembangan \textit{Emergency Oracle} (termasuk modul Resolusi Identitas).         &                                         & \cellcolor{gray!50} &                     &                     \\ \hline

    4                            & Pengembangan Client untuk menyesuaikan dengan perubahan arsitektur dan persiapan integrasi dengan Gateway, dan mekanisme \textit{break-glass}. &                                         & \cellcolor{gray!50} & \cellcolor{gray!50} &                     \\ \hline

    5                            & Integrasi Sistem (End-to-End) termasuk penggabungan modul klien, gateway, oracle, IOTA dan IPFS.                                               &                                         &                     & \cellcolor{gray!50} &                     \\ \hline

    6                            & Pengujian fungsional dan perbaikan sistem.                                                                                                     &                                         &                     & \cellcolor{gray!50} & \cellcolor{gray!50} \\ \hline

    7                            & Analisis hasil pengujian dan penulisan Laporan Tugas Akhir.                                                                                    &                                         &                     & \cellcolor{gray!50} & \cellcolor{gray!50} \\ \hline
\end{longtable}

