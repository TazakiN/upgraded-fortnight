\section{Kontrol Akses}
\label{sec:kontrol-akses}

\subsection{Definisi Kontrol Akses}
\label{subsec:definisi-kontrol-akses}
Kontrol Akses adalah proses pengaturan akses ke sumber daya (data, sistem informasi, perangkat, atau area tertentu). Proses ini memutuskan apakah seorang subjek harus diberikan atau ditolak aksesnya berdasarkan aturan tertentu. Akses dapat berarti memakai, memasuki, atau menggunakan. Kontrol Akses bertujuan untuk mencegah aktivitas yang dapat menimbulkan pelanggaran keamanan seperti kebocoran data atau modifikasi tidak sah. \autocite{sandhu1994access}.

\subsection{Model Kontrol Akses}
\label{subsec:model-kontrol-akses}
Model Kontrol Akses adalah model yang menjelaskan bagaimana kebijakan akses diterapkan dalam sistem. Setiap model memiliki kelebihan dan kekurangan masing-masing. Pemilihan model yang tepat tergantung pada kebutuhan dan konteks sistem. Ketika sebuah model dirasa tidak cukup untuk memenuhi kebutuhan, maka dapat digunakan model hybrid yang merupakan perpaduan dari beberapa model tradisional dan modern \autocite{sohofi2025literature}.

Pada setiap model kontrol akses, terdapat 3 tahapan dasar yang harus dilalui oleh seorang pengguna ketika ingin mengakses sumber daya di dalam sistem yaitu:

\begin{enumerate}
	\item \textbf{Identifikasi} \\
	      Kontrol akses diinisiasi dengan identifikasi subjek menggunakan identitas unik yang membedakan antar pengguna di dalam lingkup sistem.
	\item \textbf{Autentikasi} \\
	      Autentikasi adalah proses verifikasi identitas pengguna untuk memastikan bahwa pengguna yang mengakses sistem adalah pengguna yang valid. Metode klaim identitas bisa beragam, namun umumnya ada diantara tiga kategori berikut:
	      \begin{itemize}
		      \item \textit{\textbf{Something that you know}}: Password, PIN, atau kode otentikasi lainnya.
		      \item \textit{\textbf{Something that you have}}: Kartu identitas, ponsel cerdas, atau perangkat fisik lainnya.
		      \item \textit{\textbf{Something that you are}}: Irisan jari, sidik jari, atau fitur pengenalan wajah lainnya.
	      \end{itemize}
	\item \textbf{Otorisasi} \\
	      Otorisasi adalah proses pemberian izin atau akses kepada pengguna untuk mengakses sumber daya tertentu.
\end{enumerate}

\subsection{Model Capability-Based Access Control (CapBAC)}
\label{subsec:model-capability-based-access-control-capbac}
Model Capability-Based Access Control (CapBAC) adalah paradigma kontrol akses dimana hak akses didefinisikan melalui sebuah token yang tidak bisa diubah dan disebut kapabilitas. Token yang dimiliki oleh seorang pengguna menentukan izin atau hak yang dimiliki oleh pengguna untuk mengakses sumber daya tertentu, termasuk delegasi hak akses kepada pengguna lain, kondisi, dan batasan. Jika dibandingkan dengan model lain seperti Role-Based Access Control (RBAC) dan Attribute-Based Access Control (ABAC) yang bergantung pada sentralisasi, CapBAC memungkinkan otorisasi terdesentralisasi karena memiliki kemampuan untuk membuat bukti otorisasi yang tidak dapat dipalsukan \autocite{nakamura2020exploiting}.

Model CapBAC sudah memiliki beberapa implementasi yang telah dikembangkan seperti pada perangkat IoT \autocite{gusmeroli2013capability} dan teknologi Blockchain \autocite{liu2021capability}. Terdapat juga beberapa modifikasi yang telah dilakukan pada model CapBAC seperti Enhanced Context-Aware Capability-Based Access Control (ECACBAC) \autocite{ahamed2019enhanced} yang menambahkan konteks bagaimana akses dilakukan oleh pengguna dalam model token kapabilitas yang meningkatkan \textit{trustworthiness} dari akses, dan Traceable Capability-Based Access Control (TCAC) \autocite{li2022traceable} yang menambahkan time capability tree ke dalam model token kapabilitas. Setiap modifikasi dari model akses kontrol memiliki kelebihan dan kekurangan masing-masing, namun semua modifikasi ini memiliki tujuan yang sama yaitu untuk meningkatkan keamanan dan kemudahan dalam penggunaan model CapBAC pada konteks sistem yang dibangun.

% Fleksibilitas yang dimiliki oleh CapBAC memungkinkan penerapan konsep \textit{least privilege} yang dapat mengurangi risiko terjadinya serangan karena pengguna hanya memiliki akses ke sumber daya yang diperlukan dan tidak memiliki akses ke sumber daya yang tidak perlu.

% Integrasi teknologi blockchain dengan CapBAC menawarkan keuntungan signifikan untuk kontrol akses layanan kesehatan. \textit{Smart Contract} memungkinkan registrasi, propagasi, dan pencabutan token kemampuan secara otomatis. Pendekatan berbasis blockchain ini mengatasi titik kegagalan tunggal yang melekat pada server otorisasi terpusat sambil menyediakan jejak audit yang tidak dapat diubah untuk semua keputusan kontrol akses \autocite{xu2018blendcac}. 

Studi tentang CapBAC yang didukung blockchain menunjukkan kelayakan praktis. Kerangka kerja ZKP-CapBAC menggabungkan konsep \textit{zero-knowledge proofs} dengan kontrol akses berbasis kemampuan untuk delegasi lintas domain dalam lingkungan blockchain, memungkinkan verifikasi hak akses yang menjaga privasi. Evaluasi kinerja menunjukkan bahwa sistem RBAC berbasis blockchain dapat mencapai \textit{throughput} yang superior dan latensi yang lebih rendah dibandingkan dengan pendekatan terpusat tradisional \autocite{chen2025zkpcapbac}.