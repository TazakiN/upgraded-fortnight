\section{Latar Belakang}
\label{sec:latarbelakang}

Pada era digital ini, transformasi sistem rekam medis didorong secara kuat oleh regulasi pemerintah. Peraturan Menteri Kesehatan Republik Indonesia (Permenkes) Nomor 24 Tahun 2022 mewajibkan fasilitas layanan kesehatan (fasyankes) untuk mengimplementasikan Rekam Medis Elektronik (RME) yang terintegrasi \autocite{permenkes24_2022}. Upaya ini diperkuat dengan Keputusan Menteri Kesehatan HK.01.07/MENKES/1423/2022 tentang standar variabel RME \autocite{kepmenkes1423_2022} dan inisiatif platform SATUSEHAT yang bertujuan mengintegrasikan lebih dari 400 aplikasi RME di seluruh Indonesia \autocite{martinadewi2024analysis}. Integrasi ini krusial untuk menjamin interoperabilitas data kesehatan nasional sesuai amanat Undang-Undang Nomor 29 Tahun 2004 \autocite{uu_29_2004}.

Meskipun integrasi melalui SATUSEHAT patut didukung, arsitektur yang tersentralisasi menciptakan risiko keamanan inheren berupa \textit{Single Point of Failure} (SPoF). Pemusatan data dalam skala masif meningkatkan dampak fatal apabila terjadi kegagalan sistem atau peretasan. Sektor kesehatan tercatat sebagai sektor yang rentan terhadap serangan siber secara global \autocite{vankessel2023strengthening} \autocite{hipaajournal2024report}. Indonesia tidak luput dari ancaman ini, sebagaimana terindikasi dari dugaan kebocoran jutaan data rekam medis pasien yang dikelola Kementerian Kesehatan pada tahun 2022 \autocite{hukumonline_kemenkes}. Risiko ini memicu kebutuhan akan arsitektur alternatif yang lebih tangguh, yaitu sistem terdesentralisasi berbasis \textit{Distributed Ledger Technology} (DLT) yang dapat menghilangkan SPoF sekaligus mengembalikan kedaulatan data kepada pasien \autocite{ferreira2024enhancing}.

Namun, penerapan sistem terdesentralisasi pada sektor kesehatan memiliki tantangan tersendiri, terutama pada aspek kontrol akses (\textit{access control}) \autocite{alhaqbani2008access} \autocite{rele2023securing}. Tantangan utamanya bukan hanya menciptakan sistem yang aman, tetapi juga menyeimbangkan keamanan ketat dengan efisiensi alur kerja klinis \autocite{psarra2024permissioned} \autocite{beard2012challenges}. Penelitian menunjukkan bahwa mekanisme keamanan yang terlalu kaku sering kali menghambat tenaga medis dalam memberikan perawatan cepat, sehingga memicu frustrasi atau penundaan layanan medis yang krusial \autocite{Rajput2021}. Selain itu, kontrol akses harus dapat menjamin kerahasiaan data pasien dari penggunaan sekunder yang tidak sah, seperti komersialisasi atau penelitian tanpa persetujuan (\textit{consent}) \autocite{alhomidan2025confidentiality}.

% Kontrol akses dari sistem EHR tidak hanya sekedar memberikan akses data kepada "dokter" dan "perawat". Kontrol bersifat \textit{fine-grained} dibutuhkan untuk membatasi akses data kepada pihak tertentu.  Oleh karena itu, diperlukan mekanisme yang mampu memberikan hak akses secara granular dengan pendekatan \textit{patient-centric} \autocite{peng2023patientcentric} \autocite{leventhal2015designing}.

Model akses kontrol konvensional dinilai belum memadai untuk mengatasi kompleksitas ini. Model \textit{Role-Based Access Control} (RBAC), meskipun menjadi standar industri, dianggap terlalu statis dan kasar (\textit{coarse-grained}) karena memberikan akses luas berdasarkan jabatan tanpa mempertimbangkan konteks hubungan dokter-pasien spesifik
% TODO: sitasi rbac
. Sementara model \textit{Attribute-Based Access Control} (ABAC) menawarkan granularitas lebih baik, namun memiliki kompleksitas manajemen kebijakan yang tinggi dan beban komputasi yang berat \autocite{Pussewalage2016}. Oleh karena itu, diperlukan mekanisme yang mampu memberikan hak akses secara granular, efisien, dan mendukung prinsip \textit{patient self-sovereignty}—di mana pasien memegang kendali penuh atas data mereka \autocite{leventhal2015designing}.

Beberapa solusi terdesentralisasi telah diusulkan, salah satunya adalah DecMed. Sistem ini memanfaatkan DLT IOTA, IPFS, dan \textit{Proxy Re-Encryption} (PRE) dengan menerapkan \textit{capability-based access control} \autocite{exceline2023security}. Meskipun DecMed berhasil mengamankan data menggunakan enkripsi, mekanisme PRE yang digunakan dinilai masih memiliki keterbatasan dalam hal fleksibilitas delegasi (pembagian) akses ke pihak ketiga secara spesifik. Lebih jauh lagi, sistem ini belum optimal dalam menangani akses darurat (\textit{break-glass}) yang membutuhkan respons cepat tanpa prosedur otentikasi yang berbelit, namun tetap harus tercatat secara immutable untuk kebutuhan audit \autocite{dias2018blockchain}.

Berdasarkan permasalahan tersebut, Tugas Akhir ini mengusulkan pengembangan kerangka kerja sistem manajemen RME berbasis DecMed dengan memperbaharui mekanisme kontrol aksesnya menggunakan teknologi Macaroons \autocite{birgisson2014macaroons}. Macaroons dipilih karena kemampuannya dalam mendukung delegasi terdesentralisasi dengan caveats (persyaratan) yang kontekstual. Pengembangan ini difokuskan untuk mengatasi kelemahan mekanisme delegasi dan akses darurat pada penelitian sebelumnya, sehingga dapat meningkatkan keamanan data RME sekaligus memberikan kontrol akses yang utuh dan granular kepada pasien.