\chapter{Pendahuluan}
\label{chap:pendahuluan}

\section{Latar Belakang}
\label{sec:latarbelakang}

Pada era digital ini, transformasi sistem rekam medis didorong secara kuat oleh regulasi pemerintah. Peraturan Menteri Kesehatan Republik Indonesia (Permenkes) Nomor 24 Tahun 2022 mewajibkan fasilitas layanan kesehatan (fasyankes) untuk mengimplementasikan Rekam Medis Elektronik (RME) yang terintegrasi \autocite{permenkes24_2022}. Upaya ini diperkuat dengan Keputusan Menteri Kesehatan HK.01.07/MENKES/1423/2022 tentang standar variabel RME \autocite{kemenkes1423-2022} dan inisiatif platform SATUSEHAT yang bertujuan mengintegrasikan lebih dari 400 aplikasi RME di seluruh Indonesia \autocite{martinadewi2024analysis}. Integrasi ini krusial untuk menjamin interoperabilitas data kesehatan nasional sesuai amanat Undang-Undang Nomor 29 Tahun 2004 \autocite{uu_29_2004}.

Meskipun integrasi melalui SATUSEHAT patut didukung, arsitektur yang tersentralisasi menciptakan risiko keamanan inheren berupa \textit{Single Point of Failure} (SPoF) \autocite{pethuraj2024securing}. Pemusatan data dalam skala masif meningkatkan dampak fatal apabila terjadi kegagalan sistem atau peretasan. Sektor kesehatan tercatat sebagai sektor yang rentan terhadap serangan siber secara global \autocite{vankessel2023strengthening} \autocite{hipaajournal2024report}. Indonesia tidak luput dari ancaman ini, sebagaimana terindikasi dari dugaan kebocoran jutaan data rekam medis pasien yang dikelola Kementerian Kesehatan pada tahun 2022 \autocite{hukumonline_kemenkes}. Risiko ini memicu kebutuhan akan arsitektur alternatif yang lebih tangguh, yaitu sistem terdesentralisasi berbasis \textit{Distributed Ledger Technology} (DLT) yang dapat menghilangkan SPoF sekaligus mengembalikan kedaulatan data kepada pasien \autocite{ferreira2024enhancing}.

Namun, penerapan sistem informasi kesehatan memiliki tantangan tersendiri, terutama pada aspek kontrol akses \autocite{alhaqbani2008access} \autocite{rele2023securing}. Tantangan utamanya bukan hanya menciptakan sistem yang aman, tetapi juga menyeimbangkan keamanan ketat dengan efisiensi alur kerja klinis \autocite{psarra2024permissioned} \autocite{beard2012challenges}. Penelitian menunjukkan bahwa mekanisme keamanan yang terlalu kaku sering kali menghambat tenaga medis dalam memberikan perawatan cepat, sehingga memicu frustrasi atau penundaan layanan medis yang krusial \autocite{Rajput2021}. Selain itu, kontrol akses harus dapat menjamin kerahasiaan data pasien dari penggunaan sekunder yang tidak sah, seperti komersialisasi atau penelitian tanpa persetujuan \autocite{alhomidan2025confidentiality}.

Regulasi melalui Permenkes No. 24 Tahun 2022 Pasal 26 dan 28 secara spesifik menuntut pembatasan akses berbasis tanggung jawab \autocite{permenkes24_2022}. Namun, kontrol akses EHR konvensional seringkali hanya membedakan peran kasar seperti 'dokter' atau 'perawat', padahal kebutuhan lapangan menuntut kontrol yang bersifat \textit{fine-grained} dan \textit{patient-centric} \autocite{peng2023patientcentric} \autocite{leventhal2015designing}. Selain terkait akses pada kondisi normal, sistem EHR juga perlu mekanisme akses darurat pada kondisi pasien tidak sadar atau dalam keadaan darurat \autocite{permenkes24_2022} \autocite{leventhal2015designing}. Dalam keadaan darurat, akses darurat perlu dapat dilakukan dengan cepat, tanpa melalui prosedur otentikasi yang berbelit. Namun akses tersebut tetap tercatat secara \textit{immutable} untuk kebutuhan audit \autocite{alamin2025did}. Oleh karena itu, dibutuhkan mekanisme kontrol akses yang mampu menegakkan batasan granular pada kondisi normal, namun tetap efisien dan \textit{auditable} saat kondisi darurat medis terjadi.

Model akses kontrol konvensional dinilai belum memadai untuk mengatasi kompleksitas ini. Model \textit{Role-Based Access Control} (RBAC), meskipun menjadi standar industri, dianggap terlalu statis dan kasar (\textit{coarse-grained}) karena memberikan akses luas berdasarkan jabatan tanpa mempertimbangkan konteks hubungan dokter-pasien spesifik \autocite{Alshehri2013} \autocite{deCarvalhoJunior2018}. Meskipun model \textit{Attribute-Based Access Control} (ABAC) menawarkan granularitas lebih baik, implementasinya pada lingkungan terdesentralisasi membebani kinerja sistem akibat evaluasi kebijakan yang kompleks di setiap permintaan akses \autocite{Pussewalage2016}. Hal ini menjadi kendala fatal dalam skenario medis yang menuntut latensi rendah.

Beberapa solusi terdesentralisasi telah diusulkan, salah satunya adalah DecMed. Sistem ini memanfaatkan DLT IOTA, IPFS, dan \textit{Proxy Re-Encryption} (PRE) dengan menerapkan \textit{capability-based access control} \autocite{exceline2023security}. Meskipun DecMed berhasil mendelegasikan hak akses dari pasien kepada tenaga medis, mekanisme token kapabilitas pada DecMed bersifat statis, di mana pendelegasian akses ke pihak ketiga memerlukan penerbitan token baru oleh pemilik data. Hal ini tidak efisien untuk skenario delegasi berantai di lingkungan medis yang dinamis. Lebih jauh lagi, sistem ini belum optimal dalam menangani akses darurat (\textit{break-glass}) yang membutuhkan respons cepat tanpa prosedur otentikasi yang berbelit, namun tetap harus tercatat secara immutable untuk kebutuhan audit \autocite{dias2018blockchain}.

Berdasarkan permasalahan tersebut, Tugas Akhir ini mengusulkan pengembangan kerangka kerja sistem manajemen RME berbasis DecMed dengan memperbaharui mekanisme kontrol aksesnya menggunakan teknologi Macaroons. Kebaruan utama penelitian ini terletak pada pemanfaatan fitur \textit{attenuation} pada Macaroons \autocite{birgisson2014macaroons}. Berbeda dengan mekanisme DecMed, Macaroons memungkinkan pemegang akses untuk mendelegasikan haknya secara terbatas kepada pihak lain tanpa perlu berinteraksi kembali dengan pasien, serta mendukung verifikasi caveats kontekstual untuk skenario break-glass yang tamper-evident. Pengembangan ini difokuskan untuk mengatasi kelemahan mekanisme delegasi dan akses darurat pada penelitian sebelumnya, sehingga dapat meningkatkan keamanan data RME sekaligus memberikan kontrol akses yang utuh dan granular kepada pasien.

\section{Rumusan Masalah}

Berdasarkan latar belakang yang telah diuraikan pada subbab \ref{sec:latarbelakang}, maka rumusan masalah yang dibahas dalam tugas akhir ini adalah sebagai berikut:

\begin{enumerate}
	\item Bagaimana merancang mekanisme kontrol akses granular pada arsitektur RME terdesentralisasi (DecMed) dengan memanfaatkan atribut attenuation pada Macaroons?
	\item Bagaimana mengembangkan mekanisme delegasi RME untuk kasus gawat darurat?
\end{enumerate}

\section{Tujuan}

Tujuan dari Tugas Akhir ini adalah mengimplementasikan kerangka kerja kontrol akses pada sistem manajemen Rekam Medis Elektronik (RME) berbasis arsitektur DecMed dengan memanfaatkan teknologi Macaroons. Pengembangan ini difokuskan untuk memfasilitasi delegasi hak akses yang bersifat granular serta menyediakan mekanisme akses gawat darurat (\textit{break-glass}) tanpa prosedur otentikasi yang kompleks, dengan tetap menjamin integritas data dan pencatatan audit yang \textit{immutable}.

\section{Batasan Masalah}

Terdapat beberapa batasan masalah dari Tugas Akhir ini. Batasan masalah tersebut
antara lain:

\begin{enumerate}
	\item Sistem manajemen RME dirancang sebagai pengembangan dari arsitektur DecMed yang memanfaatkan DLT IOTA dan IPFS;
	\item Fokus penelitian pada mekanisme otorisasi dan delegasi akses; penanganan keamanan terkait kehilangan kredensial pengguna berada di luar lingkup pembahasan;
	\item Validasi pada mekanisme \textit{break-glass} dibatasi pada validasi IP jaringan dari fasyankes; verifikasi kondisi medis aktual pasien berada di luar lingkup sistem.
	\item Pada keadaan darurat, diasumsikan bahwa data NIK diketahui oleh tenaga medis yang meminta akses darurat.
\end{enumerate}

\section{Metodologi}

Metodologi yang digunakan dalam Tugas Akhir ini terdiri dari empat tahap pelaksanaan, sebagai berikut:

\begin{enumerate}
	\item \textbf{Studi Literatur dan Analisis Kebutuhan}
	      \label{subsec:studi-literatur-dan-analisis-kebutuhan}

	      Tahap awal ini berfokus pada pengumpulan fondasi teoretis dan kontekstual melalui studi literatur mendalam mengenai arsitektur DecMed, teknologi DLT (khususnya IOTA), model CapBAC, dan implementasi Macaroons untuk manajemen akses. Selain itu, dilakukan analisis komparatif terhadap penelitian terkait untuk mengidentifikasi kelemahan dan peluang pengembangan dari solusi yang sudah ada, serta analisis regulasi, khususnya Permenkes No. 24 Tahun 2022, untuk memastikan kerangka kerja yang diusulkan sejalan dengan kebutuhan hukum di Indonesia. Tahap ini diakhiri dengan perumusan kebutuhan fungsional dan non-fungsional sistem berdasarkan hasil studi dan analisis tersebut.

	\item \textbf{Perancangan Arsitektur dan Model Kontrol Akses}
	      \label{subsec:perancangan-arsitektur-dan-model-kontrol-akses}

	      Berdasarkan analisis kebutuhan, tahap selanjutnya berfokus pada perancangan teknis sistem, yang dimulai dengan merancang arsitektur sistem RME terdesentralisasi yang mengadaptasi dan mengembangkan arsitektur DecMed. Perancangan ini juga mencakup model kontrol akses berbasis Macaroons secara rinci, termasuk mendefinisikan struktur \textit{caveats} (batasan) untuk granularitas akses, serta mendesain alur kerja (\textit{workflow}) untuk proses-proses kunci seperti penerbitan (\textit{minting}) kapabilitas, atenuasi (\textit{attenuation}) oleh pasien, dan verifikasi kapabilitas oleh pihak ketiga. Selain itu, dirancang pula skema khusus untuk mekanisme delegasi RME dalam skenario gawat darurat.

	\item \textbf{Implementasi Prototipe}
	      \label{subsec:implementasi-prototipe}

	      Tahap ini merupakan realisasi dari rancangan ke dalam bentuk prototipe fungsional. Kegiatan dimulai dengan konfigurasi lingkungan pengembangan, termasuk penyiapan \textit{node} IOTA (atau simulasinya) dan sistem penyimpanan data terdistribusi (IPFS). Selanjutnya dilakukan pengembangan \textit{services} inti yang mencakup logika untuk membuat, mengatenuasi, dan memverifikasi Macaroons, serta implementasi \textit{business logic} untuk menangani skenario akses normal (misalnya, pasien ke dokter) dan skenario akses gawat darurat. Seluruh komponen kemudian diintegrasikan menjadi satu \textit{framework} prototipe yang kohesif.

	\item \textbf{Pengujian dan Evaluasi Fungsional}
	      \label{subsec:pengujian-dan-evaluasi-fungsional}

	      Tahap akhir bertujuan untuk memvalidasi fungsionalitas sistem melalui pengujian fungsional guna memverifikasi bahwa semua alur kerja yang dirancang berjalan sesuai ekspektasi. Simulasi berbagai skenario akses dilakukan dengan fokus utama pada pengujian mekanisme delegasi gawat darurat untuk memvalidasi efektivitas dan keamanan model kontrol akses. Hasil pengujian kemudian dianalisis untuk menilai keberhasilan prototipe dalam mengatasi tantangan delegasi akses granular dan akses darurat, serta pemenuhan tujuan penelitian.

\end{enumerate}

% \begin{enumerate}
% 	\item \textbf{Tahap Perencanaan} \\
% 	      Tahap ini berfokus pada perencanaan implementasi dari Capability-based Access Control pada DecMed dengan memanfaatkan teknologi Macaroons dan perencanaan mekanisme delegasi RME untuk kasus gawat darurat. Kegiatan utama pada tahap ini me  liputi analisis komparatif penelitian-penelitian yang ada dan analisis terhadap regulasi yang berlaku saat ini.
% 	\item \textbf{Tahap Perancangan} \\
% 	      Setelah dilakukan perencanaan pada tahap sebelumnya, pada tahap ini akan dilakukan perancangan dari sistem yang dikembangkan. Perancangan dimulai dari menentukan kebutuhan fungsional dan nonfungsional sistem. Setelah seluruh kebutuhan telah ditentukan kemudian akan dilanjutkan dengan merancang arsitektur dan alur kerja dari sistem. Rancangan tersebut selanjutnya digunakan pada tahap implementasi.
% 	\item \textbf{Tahap Implementasi} \\
% 	      Pada tahap ini, kerangka kerja yang telah dirancang pada tahap sebelumnya akan diimplementasikan dalam bentuk nyata. Proses implementasi mencakup konfigurasi service-service yang digunakan serta implementasi business logic sistem. Tahap ini juga memastikan kerangka kerja dapat diimplementasikan secara fungsional sesuai dengan rancangan yang telah dibuat sebelumnya.
% 	\item \textbf{Tahap Pengujian dan Simulasi} \\
% 	      Tahap terakhir bertujuan untuk menguji dan memvalidasi kerangka kerja yang telah diimplementasikan. Pengujian meliputi evaluasi fungsionalitas sistem, termasuk menguji masing-masing komponen dan sistem secara keseluruhan. Proses ini bertujuan untuk memastikan kerangka kerja yang dirancang mampu memenuhi kebutuhan fungsional dan nonfungsional yang telah didefinisikan serta memenuhi standar yang diharapkan.

% \end{enumerate}

\section{Sistematika Pembahasan}
\label{sec:sistematika-pembahasan}

Berikut merupakan sistematika pembahasan dari Tugas Akhir ini:

\begin{description}
	\item[Bab I - Pendahuluan]
	      Bab ini menjelaskan gagasan utama dari tugas akhir, yang meliputi latar belakang, rumusan masalah, tujuan, batasan, metodologi, hingga sistematika pembahasan mengenai proses pengembangan sistem manajemen RME.

	\item[Bab II - Studi Literatur]
	      Bab ini memaparkan studi literatur yang telah dilakukan terkait sistem yang dikembangkan. Pembahasan difokuskan pada uraian teoritis berbagai komponen yang digunakan, termasuk regulasi, algoritma, dan teknologi yang digunakan dalam penyusunan sistem.

	\item[Bab III - Analisis dan Perancangan Sistem]
	      Pada bab ini dijelaskan analisis terhadap masalah yang ditemukan serta berbagai alternatif solusi yang dapat digunakan untuk mengatasinya. Selain itu, bab ini juga membahas rancangan sistem manajemen RME yang dikembangkan, meliputi pendefinisian kebutuhan fungsional dan nonfungsional, serta rancangan setiap komponen sistem.


	\item[Bab IV - Rencana Pelaksanaan]
	      Bab ini menjelaskan rencana kerja untuk merealisasikan solusi yang telah dirancang. Pembahasan mencakup jadwal pelaksanaan Tugas Akhir yang disusun per tahap kegiatan, serta identifikasi risiko-risiko teknis maupun non-teknis yang mungkin dihadapi selama pengerjaan beserta rencana mitigasinya.

	      % \item \textbf{Bab IV - Implementasi dan Pengujian Sistem} \\
	      %       Bab ini menjelaskan hasil implementasi dari rancangan sistem yang telah dipaparkan pada Bab III. Selain itu, dibahas pula mekanisme pengujian serta hasil pengujian yang dilakukan terhadap sistem manajemen RME.

	      % \item \textbf{Bab V - Penutup} \\
	      %       Bab terakhir ini berisi kesimpulan dan saran setelah proses perancangan, implementasi, dan pengujian sistem. Penulis berharap laporan ini dapat memberikan edukasi serta menjadi pijakan awal bagi penelitian atau pengembangan selanjutnya.
\end{description}