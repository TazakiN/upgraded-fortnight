\section{Prosedur Break Glass}
\label{sec:prosedur-break-glass}

Mekanisme \textit{break-glass} mengatasi ketegangan kritis dalam kontrol akses layanan kesehatan: mempertahankan \textit{protokol} keamanan yang ketat sambil memungkinkan kemampuan \textit{override} darurat ketika perawatan pasien memerlukan akses segera ke rekam medis. Mekanisme ini berakar pada prinsip layanan kesehatan bahwa perawatan pasien lebih diutamakan daripada pembatasan kontrol akses dalam situasi yang mengancam jiwa \autocite{yeng2022analysing}.

Konsep akses \textit{break-glass} berasal dari mekanisme keamanan fisik di mana kaca harus dipecahkan untuk mengakses peralatan darurat, meninggalkan bukti penggunaan yang terlihat. Dalam \textit{sistem} layanan kesehatan digital, ini diterjemahkan menjadi mekanisme \textit{override} yang: (1) memberikan akses segera kepada personel darurat yang berwenang, (2) membuat \textit{jejak audit} yang tidak dapat diubah, dan (3) memicu proses peninjauan pasca-kejadian.

Beberapa pendekatan arsitektur untuk mekanisme \textit{break-glass} telah diusulkan. \textit{Sistem} Kontrol Akses \textit{Break-Glass} Ringan (\textit{Lightweight Break-Glass Access Control/LiBAC}) mengimplementasikan \textit{mode} akses ganda: akses berbasis \textit{atribut} untuk situasi normal dan akses \textit{break-glass} yang melewati kebijakan akses selama keadaan darurat \autocite{yang2018lightweight}. \textit{Sistem} ini terbukti aman secara formal dalam model standar sambil mempertahankan efisiensi yang sesuai untuk \textit{jaringan IoT} layanan kesehatan.