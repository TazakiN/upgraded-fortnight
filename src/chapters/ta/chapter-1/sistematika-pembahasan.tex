\section{Sistematika Pembahasan}
\label{sec:sistematika-pembahasan}

Berikut merupakan sistematika pembahasan dari Tugas Akhir ini:

\begin{description}
	\item[Bab I - Pendahuluan]
	      Bab ini menjelaskan gagasan utama dari tugas akhir, yang meliputi latar belakang, rumusan masalah, tujuan, batasan, metodologi, hingga sistematika pembahasan mengenai proses pengembangan sistem manajemen RME.

	\item[Bab II - Studi Literatur]
	      Bab ini memaparkan studi literatur yang telah dilakukan terkait sistem yang dikembangkan. Pembahasan difokuskan pada uraian teoritis berbagai komponen yang digunakan, termasuk regulasi, algoritma, dan teknologi yang digunakan dalam penyusunan sistem.

	\item[Bab III - Analisis dan Perancangan Sistem]
	      Pada bab ini dijelaskan analisis terhadap masalah yang ditemukan serta berbagai alternatif solusi yang dapat digunakan untuk mengatasinya. Selain itu, bab ini juga membahas rancangan sistem manajemen RME yang dikembangkan, meliputi pendefinisian kebutuhan fungsional dan nonfungsional, serta rancangan setiap komponen sistem.


	\item[Bab IV - Rencana Pelaksanaan]
	      Bab ini menjelaskan rencana kerja untuk merealisasikan solusi yang telah dirancang. Pembahasan mencakup jadwal pelaksanaan Tugas Akhir yang disusun per tahap kegiatan, serta identifikasi risiko-risiko teknis maupun non-teknis yang mungkin dihadapi selama pengerjaan beserta rencana mitigasinya.

	      % \item \textbf{Bab IV - Implementasi dan Pengujian Sistem} \\
	      %       Bab ini menjelaskan hasil implementasi dari rancangan sistem yang telah dipaparkan pada Bab III. Selain itu, dibahas pula mekanisme pengujian serta hasil pengujian yang dilakukan terhadap sistem manajemen RME.

	      % \item \textbf{Bab V - Penutup} \\
	      %       Bab terakhir ini berisi kesimpulan dan saran setelah proses perancangan, implementasi, dan pengujian sistem. Penulis berharap laporan ini dapat memberikan edukasi serta menjadi pijakan awal bagi penelitian atau pengembangan selanjutnya.
\end{description}