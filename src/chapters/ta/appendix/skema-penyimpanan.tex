\chapter{Skema Penyimpanan Data \textit{On-Chain} dan \textit{Off-Chain}}
\label{lampiran:skema-penyimpanan}

Data rekam medis sensitif disimpan secara \textit{Off-Chain} (IPFS) dalam bentuk terenkripsi (\textit{Encrypted Blob}), sedangkan metadata kontrol akses, kapsul kunci (\textit{Key Capsules}), dan log audit disimpan secara \textit{On-Chain} pada IOTA Tangle.

\section{Skema Data Off-Chain (IPFS)}
\label{sec:skema-offchain}

Tabel berikut menjabarkan struktur data JSON yang dienkripsi sebelum diunggah ke IPFS.

% TABEL 1: DATA ADMINISTRATIF
\begin{longtable}{|>{\raggedright\arraybackslash}p{3.0cm}|>{\raggedright\arraybackslash}p{4.5cm}|>{\raggedright\arraybackslash}p{5.5cm}|}
    \caption{Skema Data Administratif (Enkripsi Konten)} \label{tab:offchain-admin}   \\
    \hline
    \textbf{Komponen} & \textbf{Key JSON}                       & \textbf{Keterangan} \\
    \hline
    \endfirsthead
    \hline
    \textbf{Komponen} & \textbf{Key JSON}                       & \textbf{Keterangan} \\
    \hline
    \endhead

    % METADATA
    \multirow{3}{3.0cm}{\textbf{Header \& Metadata}}
                      & \texttt{record\_id}
                      & UUID unik (v4) sebagai identitas file.                        \\ \cline{2-3}
                      & \texttt{dataset\_category}
                      & Bernilai \texttt{"ADMINISTRATIVE"}.                           \\ \cline{2-3}
                      & \texttt{patient\_id}
                      & Hash NIK Pasien (Identitas DecMed).                           \\ \hline

    % PAYLOAD DATA
    \multirow{5}{3.0cm}{\textbf{Payload Data}}
                      & \texttt{full\_name}
                      & Nama lengkap sesuai KTP.                                      \\ \cline{2-3}
                      & \texttt{nik}
                      & Nomor Induk Kependudukan (16 digit).                          \\ \cline{2-3}
                      & \texttt{dob}
                      & Tanggal lahir (format YYYY-MM-DD).                            \\ \cline{2-3}
                      & \texttt{insurance\_type}
                      & Jenis penjamin (BPJS/Pribadi/Asuransi).                       \\ \cline{2-3}
                      & \texttt{emergency\_contact}
                      & Nomor kontak darurat keluarga terdekat.                       \\ \hline
\end{longtable}

\vspace{0.5cm}

% TABEL 2: DATA KLINIS
\begin{longtable}{|>{\raggedright\arraybackslash}p{3.0cm}|>{\raggedright\arraybackslash}p{4.5cm}|>{\raggedright\arraybackslash}p{5.5cm}|}
    \caption{Skema Data Klinis (Contoh: Rawat Jalan)} \label{tab:offchain-clinical}                                                                      \\
    \hline
    \textbf{Komponen} & \textbf{Key JSON}                                                                                          & \textbf{Keterangan} \\
    \hline
    \endfirsthead
    \hline
    \textbf{Komponen} & \textbf{Key JSON}                                                                                          & \textbf{Keterangan} \\
    \hline
    \endhead

    % METADATA
    \multirow{5}{3.0cm}{\textbf{Header \& Metadata}}
                      & \texttt{record\_id}
                      & UUID unik file ini.                                                                                                              \\ \cline{2-3}
                      & \texttt{dataset\_category}
                      & Kategori unit, misalnya \texttt{IGD}, \texttt{RAWAT\_JALAN}. Merujuk pada tabel \ref{tab:mapping-dataset}.                       \\ \cline{2-3}
                      & \texttt{visit\_date}
                      & Tanggal kunjungan/pemeriksaan.                                                                                                   \\ \cline{2-3}
                      & \texttt{author\_address}
                      & Alamat Wallet IOTA Dokter pembuat data.                                                                                          \\ \cline{2-3}
                      & \texttt{related\_admin\_id}
                      & Referensi ke \texttt{record\_id} data administratif.                                                                             \\ \hline

    % PAYLOAD DATA
    \multirow{4}{3.0cm}{\textbf{Payload Data}}
                      & \texttt{anamnesis}
                      & Teks keluhan utama dan riwayat penyakit.                                                                                         \\ \cline{2-3}
                      & \texttt{vital\_signs}
                      & Tekanan darah, suhu, nadi.                                                                                                       \\ \cline{2-3}
                      & \texttt{diagnosis}
                      & Array kode ICD-10 dan deskripsi.                                                                                                 \\ \cline{2-3}
                      & \texttt{medication}
                      & Array resep obat.                                                                                                                \\ \hline

    % ATTACHMENTS
    \multirow{2}{3.0cm}{\textbf{Lampiran}}
                      & \texttt{cid}
                      & Hash IPFS untuk file lampiran.                                                                                                   \\ \cline{2-3}
                      & \texttt{file\_name}
                      & Nama asli file lampiran.                                                                                                         \\ \hline
\end{longtable}

\newpage

\section{Skema Data \textit{On-Chain} (IOTA Metadata)}
\label{sec:skema-onchain}

Metadata ini disimpan secara publik di IOTA Tangle untuk keperluan penemuan data (\textit{discovery}), verifikasi integritas, dan manajemen kunci (\textit{Dual Capsule}).

\begin{longtable}{|>{\raggedright\arraybackslash}p{3.0cm}|>{\raggedright\arraybackslash}p{4.5cm}|>{\raggedright\arraybackslash}p{5.5cm}|}
    \caption{Skema Metadata \textit{On-Chain}} \label{tab:onchain-schema}                                                                              \\
    \hline
    \textbf{Komponen} & \textbf{Key JSON}                                                                                        & \textbf{Keterangan} \\
    \hline
    \endfirsthead
    \hline
    \textbf{Komponen} & \textbf{Key JSON}                                                                                        & \textbf{Keterangan} \\
    \hline
    \endhead

    \multirow{2}{3.0cm}{\textbf{Header Event}}
                      & \texttt{timestamp}
                      & Waktu presisi (Unix Epoch).                                                                                                    \\ \cline{2-3}
                      & \texttt{event}
                      & Jenis aksi (misal: \texttt{RECORD\_UPLOAD}).                                                                                   \\ \hline

    \multirow{2}{3.0cm}{\textbf{Identitas}}
                      & \texttt{patient\_address}
                      & Alamat Wallet Pasien (Hash NIK).                                                                                               \\ \cline{2-3}
                      & \texttt{fasyankes\_id}
                      & ID Instansi Fasyankes pengunggah.                                                                                              \\ \hline

    \multirow{3}{3.0cm}{\textbf{ACL Tags}}
                      & \texttt{dataset\_category}
                      & Tag dataset, misalnya \texttt{IGD}, \texttt{RAWAT\_JALAN}. Merujuk pada tabel \ref{tab:mapping-dataset}.                       \\ \cline{2-3}
                      & \texttt{visit\_date}
                      & Tanggal kunjungan (format YYYY-MM-DD).                                                                                         \\ \hline

    \multirow{2}{3.0cm}{\textbf{Pointer Data}}
                      & \texttt{ipfs\_cid}
                      & Penunjuk lokasi file terenkripsi di IPFS.                                                                                      \\ \cline{2-3}
                      & \texttt{integrity\_hash}
                      & Hash SHA-256 data asli.                                                                                                        \\ \hline

    \multirow{4}{3.0cm}{\textbf{Manajemen Kunci (Dual Capsule)}}
                      & \texttt{algo}
                      & Algoritma enkripsi (misal: \texttt{AES-256-GCM}).                                                                              \\ \cline{2-3}
                      & \texttt{capsules.normal}
                      & Kunci AES terenkripsi dengan \textbf{Public Key Pasien}.                                                                       \\ \cline{2-3}
                      & \texttt{capsules.emerg}
                      & Kunci AES terenkripsi dengan \textbf{Public Key Oracle}.                                                                       \\ \hline
\end{longtable}

\section{Skema Log Akses Darurat}
\label{sec:skema-log-darurat}

Struktur log yang dicatat oleh \textit{Emergency Oracle} saat mekanisme \textit{Break-Glass} dijalankan menggunakan skema \textit{Threshold}.

\begin{longtable}{|>{\raggedright\arraybackslash}p{3.0cm}|>{\raggedright\arraybackslash}p{4.5cm}|>{\raggedright\arraybackslash}p{5.5cm}|}
    \caption{Skema Log Audit Akses Darurat} \label{tab:onchain-emergency}                      \\
    \hline
    \textbf{Komponen} & \textbf{Key JSON}                                & \textbf{Keterangan} \\
    \hline
    \endfirsthead
    \hline
    \textbf{Komponen} & \textbf{Key JSON}                                & \textbf{Keterangan} \\
    \hline
    \endhead

    \multirow{2}{3.0cm}{\textbf{Header Event}}
                      & \texttt{event}
                      & \texttt{"EMERGENCY\_BREAK\_GLASS"}.                                    \\ \cline{2-3}
                      & \texttt{timestamp}
                      & Waktu konsensus tercapai.                                              \\ \hline

    \multirow{3}{3.0cm}{\textbf{Aktor}}
                      & \texttt{requestor\_address}
                      & Wallet Address Dokter pemohon.                                         \\ \cline{2-3}
                      & \texttt{patient\_id}
                      & ID Pasien target.                                                      \\ \hline

    \multirow{3}{3.0cm}{\textbf{Konteks}}
                      & \texttt{location\_tag}
                      & Lokasi fisik (misal: \texttt{"IGD Boromeus"}).                         \\ \cline{2-3}
                      & \texttt{reason}
                      & Alasan medis (misal: \texttt{"Cardiac Arrest"}).                       \\ \cline{2-3}
                      & \texttt{discharge\_id}
                      & ID Token unik persetujuan Oracle.                                      \\ \hline

    \multirow{3}{3.0cm}{\textbf{Konsensus}}
                      & \texttt{participating\_nodes}
                      & Array ID node Oracle yang menyetujui.                                  \\ \cline{2-3}
                      & \texttt{threshold\_status}
                      & Status kuorum (misal: \texttt{"3\_OF\_5"}).                            \\ \cline{2-3}
                      & \texttt{verdict}
                      & Keputusan (\texttt{"ACCESS\_GRANTED"}).                                \\ \hline
\end{longtable}

\section{Tabel Pemetaan Kategori Data}
\label{sec:mapping-dataset}

\begin{table}[!ht]
    \centering
    \caption{Pemetaan \textit{Dataset} untuk Kontrol Akses}
    \label{tab:mapping-dataset}
    \begin{tabular}{|>{\raggedright\arraybackslash}p{2.8cm}|>{\raggedright\arraybackslash}p{4.2cm}|>{\raggedright\arraybackslash}p{5.5cm}|}
        \hline
        \textbf{Kelompok Data KMK} & \textbf{Tag \texttt{dataset\_category}} & \textbf{Contoh Isi Variabel}        \\ \hline
        Data Administratif         & \texttt{ADMINISTRATIVE}                 & NIK, Nama, Penanggung Jawab         \\ \hline
        IGD                        & \texttt{IGD}                            & Triase, Kondisi Awal, Tindakan Cito \\ \hline
        Rawat Jalan                & \texttt{RAWAT\_JALAN}                   & Anamnesis Poli, Diagnosis Utama     \\ \hline
        Rawat Inap                 & \texttt{RAWAT\_INAP}                    & CPPT, Discharge Summary             \\ \hline
        Laboratorium               & \texttt{LABORATORIUM}                   & Hasil Lab, Gambar Rontgen           \\ \hline
        Apotek                     & \texttt{APOTEK}                         & Daftar Obat, Telaah Resep           \\ \hline
    \end{tabular}
\end{table}