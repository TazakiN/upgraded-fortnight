\section{\textit{Shamir's Secret Sharing} (SSS)}
\label{sec:shamir-secret-sharing}

\textit{Shamir's Secret Sharing} (SSS) merupakan skema berbagi rahasia yang efisien untuk mendistribusikan informasi rahasia di antara sekelompok pihak. Dalam skema ini, rahasia hanya dapat direkonstruksi jika jumlah minimum bagian-bagian tersebut digabungkan \autocite{shamir1979share}. Skema ini pertama kali dikembangkan oleh Adi Shamir, menggeneralisasi masalah manajemen kunci kriptografi dengan pendekatan \(k, n\) \textit{threshold scheme} berbasis interpolasi polinomial Lagrange. Pada rancangannya, rahasia \(S\) sebagai koefisien \(a_0\) pada polinomial \(f(x) = a_0 + a_1 x + \dots + a_{k-1} x^{k-1} \mod p\) (dengan \(p > n, S\)) dibagi menjadi \(n\) bagian \(D_i = f(i) \mod p\), dan rekonstruksi dilakukan dari minimal \(k\) titik untuk mendapatkan \(f(0) = S\) dengan keamanan yang secara teoritis sudah sempurna.

Skema ini unggul karena ukuran bagian sama dengan rahasia asli, \textit{extensible}, dan \textit{dynamic} \autocite{shamir1979share}. SSS banyak diaplikasikan untuk manajemen kunci enkripsi, \textit{wallet} kripto, \textit{multi-party computation}, serta \textit{threshold signatures}.